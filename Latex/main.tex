%\documentclass[printmode,oneside,eng]{mgr}  % do publikacji elektronicznej
\documentclass[printmode,eng]{mgr}  % do wydruku dwustronnego
%\documentclass[printmode,eng,draft]{mgr}  % do wydruku dwustronnego w~wersji roboczej

%----------------------| Input encoding configuration |

\usepackage[utf8]{inputenc}

%-------------------------| Typeface configuration |------------------------

%% Najpierw (chyba) należy wybrać czcionkę tekstu, potem matematyki

%% wybór czcionki Computer Concrete specjalnie zaprojektowanej do użycia 
%% z~czcionką matematyczną euler (nie ma ona niestety wersji bold - proponuje się więc użycie bolda z kroju
%%   Computer Modern Sans Serif)
%\usepackage{beton}             % ZALECANA DO SKŁADU TEKSTÓW PO ANGIELSKU
%\renewcommand{\bfdefault}{sbc} % to use Computer Modern Sans Serif demibold condensed fonts as bold 
%% choć może komuś bardziej podobać się będzie ten krój normalnej szerokości 
%\renewcommand{\bfdefault}{sb} % to use Computer Modern Sans Serif demibold fonts as bold 

%% alternatywny wybór Antykwy Półtawskiego - czcionki zaprojektowanej
%% specjalnie dla języka polskiego uwzględniającej jego rytm 
%\usepackage{antpolt}
%% alternatywny wybór Antykwy Toruńskiej - bardziej "współczesnej",
%% całkowicie polskiej czcionki
\usepackage{anttor}             % ZALECANA DO SKŁADU TEKSTÓW PO POLSKU
%\usepackage[math]{anttor}      % math włącza antykwę także w~matematyce
                                % ale chyba coś psuje :( (np. brak strzałek)

%% alternatywnie wybór czcionki URW Palladio
%\usepackage{newpxtext}  % Palatino font
%\linespread{1.05}  % Palatino needs more leading space (between lines)

%% ustawienie czcionki eulerowskiej do składu wyrażeń matematycznych
%\usepackage[euler-digits,small]{eulervm}
\usepackage{eulervm}

%% ustawienie czcionki bezszeryfowej Monospace (typewriter, code) font (last) - skład poleceń, wydruków programów
\usepackage[varqu,varl]{inconsolata}

%-------------------------| Sprawy polskie |------------------------

\usepackage[T1]{fontenc} % bez tego są złe znaki / { } w czcionce tt
                         % ale musi być użyty pakiet polski bez opcji
                         % wybierającej układ - gdy zamarkowane
                         % używamy pakiet polski z~opcją wyboru układu
\usepackage{polski}
%\usepackage[OT4]{polski} % domyślnie?

%---------------------------| Packetology |-------------------------

\usepackage{geometry}           % manipulowanie geometrią łamu
\usepackage{indentfirst}        % wcięcia akapitowe w pierwszych paragrafach rozdziałów
\usepackage[dvipsnames]{xcolor} % by mieć nazwy kolorów w~rodzaju \green
\usepackage{gensymb}
%% zmiana formatowania tytulariów
%\usepackage[raggedleft]{titlesec}

\usepackage[titles]{tocloft}    % do formatowani spisów

%% w razie potrzeby zmieniamy nieco odstępy przed rozdziałami i podrozdziałami w spisie
%% treści by uniknąć pojedynczego samotnego tytułu rozdziału/podrozdziału na dole/górze strony
%% i by ładnie się zmieściła notka o latechu
\addtolength{\cftbeforechapskip}{-0.9ex}
\addtolength{\cftbeforesecskip}{0.02ex}

%%modyfikacja odstępów między pozycjami w~bibliografii
\let\oldbibliography\thebibliography
\renewcommand{\thebibliography}[1]{%
  \oldbibliography{#1}%
  \setlength{\itemsep}{0pt plus 0.3ex}%
}

%% ustawienie maksymalnej liczby i obszaru dla obiektów pływających (rysunków, tabel)
\setcounter{topnumber}{3}
\setcounter{totalnumber}{4}
\renewcommand{\topfraction}{.8}

\usepackage{graphicx}                % dołaczanie i manipulowanie grafikami
\usepackage[export]{adjustbox}       % więcej komend operujących na "pudełkach"
\usepackage[caption = false]{subfig} % obsługa rysunków z częściami

\usepackage{svg}                % dołączanie grafik w formacie svg
%\usepackage[inkscapepath=svgdir]{svg}  % w nowszych wersjach

\usepackage{tikz}               % dołączanie grafik w formacie TikZ
\usepackage{makecell}           % trochę tikzowej magii
    \tikzstyle{block} = [draw, fill=blue!20, rectangle, 
     minimum height=3em, minimum width=6em]
    \tikzstyle{sum} = [draw, fill=blue!20, circle, node distance=3.5cm]
    \tikzstyle{input} = [coordinate]
    \tikzstyle{output} = [coordinate]
    \tikzstyle{pinstyle} = [pin edge={to-,thin,black}]    
    \usetikzlibrary{arrows,automata,calc,positioning}
\usepackage{standalone}  

\usepackage{pgfplots}           % do robienia wykresów 
\pgfplotsset{compat=1.5}        %% why 1.5? pl. update

\usepackage{mathtools}          % doskonałe pakiety rozszerzające do matematyki
\usepackage{amssymb, amsfonts}  % mathtools zastępuje amsmath (naprawione błedy, dodane rozszerzenia) 
\usepackage{wasysym}            % trochę więcej różnych symboli (buźki)
%\usepackage[fleqn]{mathtools}  % równania w wersji dosuniętej w lewo
%% automatyczne numerowanie jedynie tych równań, do których są odwołania w~tekście
%\mathtoolsset{showonlyrefs=true}

%% symbole do oznaczania stopek w~miejsce liczb
\renewcommand{\thefootnote}{\fnsymbol{footnote}}
%% powtórzone z latex.ltx - w przeciwnym razie znika symbol \textbardbl przy
%% wybranej Antykwie Toruńskiej (sprawdzić co z innymi definicjami z omsenc.def,
%% sprawdzić czy też przy innych czcionkach)
\DeclareTextSymbolDefault{\textbardbl}{OMS}
%% automatyczne resetowanie wartości licznika stopek użyteczne przy użyciu symboli
%% do oznaczania stopek w~miejsce liczb (liczba dostępnych symboli wynosi tylko 9)
\usepackage{etoolbox,pdftexcmds}
\makeatletter
\patchcmd{\footnote}
  {\stepcounter\@mpfn}
%  {\stepcounter\@mpfn\check@overflow\@mpfn} %%było w~przykładzie
  {\stepcounter\@mpfn\check@overflow}
  {}{}
\newcommand{\check@overflow}{%
  \ifnum\pdf@strcmp{\@mpfn}{footnote}=\z@
    \ifnum\value{footnote}>8  %tu było 9, ale czasami nie resetowało poprawnie
      \setcounter{footnote}{1}%
    \fi
  \fi
}
\makeatother

\usepackage{hyperref}           % obsługa aktywnych odnośników i hypertekstu
\usepackage{url}                % obsługa adresów url

%% pakiet minted do wydruków programów
%% 'minted' gives much better highlight then the default listings,
%% but it reqires Python with Pygments (check the documentation of minted)
%% it also requires passing '-shell-escape' option to pdflatex during compilation
\usepackage[%
  cache=false,
  chapter,
  %    outputdir=build,  % if building in separate directory, this must be included
                         % ale nie zawsze działa poprawnie :( 
    newfloat  % required if multi-page floating listings are needed (see below)
]{minted}  % (note: from my experience must be loaded before 'csquotes')
%% by wybrać inny styl - lista dostępnych styli: pygmentize -L styles
%\usemintedstyle{igor}
\SetupFloatingEnvironment{listing}{name=Wydruk}
%% kolor tła używany w~wydrukach
\definecolor{OurListingBackground}{rgb}{0.95,0.95,0.95}
%% fix the minted@colorbg environment bug
\makeatletter
\renewenvironment{minted@colorbg}[1]
 {\def\minted@bgcol{#1}%
  \noindent
  \begin{lrbox}{\minted@bgbox}
  \begin{minipage}{\linewidth-2\fboxsep}}
 {\end{minipage}%
  \end{lrbox}%
  \setlength{\topsep}{\bigskipamount}% set the vertical space
  \trivlist\item\relax % ensure going to a new line
  \colorbox{\minted@bgcol}{\usebox{\minted@bgbox}}%
  \endtrivlist % close the trivlist
 }
\makeatother

%%pakiet do robienia notatek w trakcie pracy
\usepackage{todonotes}
\makeatletter   %spolszczenie
\renewcommand{\@todonotes@todolistname}{Do zrobienia}
\renewcommand{\@todonotes@MissingFigureText}{Rysunek}
\renewcommand{\@todonotes@MissingFigureUp}{Brakujący}
\renewcommand{\@todonotes@MissingFigureDown}{rysunek}
\makeatother

%% dodaje w pliku wynikowym klucze etykiet i referencji - wygodne przy pracy nad tekstem
%\usepackage{showkeys}

%%inne przydatne
%\usepackage{fancyhdr}
%\usepackage{fancyvrb}
%\usepackage{lipsum}  
%\usepackage{listings}

%-----------------------| End of packetology |----------------------

%---------------------------| Tytularia |---------------------------

% Zmień dane stosownie do tytułu i autora pracy tutaj
% ALE TEŻ I W POLECENIU \pdftitle NIECO PONIŻEJ! (Hyper data general configuration)
\author{Maciej Gaik}
\title{Projekt budowy odwróconego wahadła matematycznego}
\engtitle{Design and construction reversed pendulum}
\supervisor{Dr inż. Mateusz Cholewiński,\\ Katedra Cybernetyki i~Robotyki}
\field{Automatyka i~Robotyka (AIR)}
\specialisation{Robotyka (ARR)}
\date{2020}

%-------------------------| Hyper data general configuration |------------------------

\hypersetup{unicode,
   pdfpagemode=UseOutlines,   % otwiera dokument w trybie jednej strony
   pdfpagelayout=SinglePage,  %
   pdfstartpage=1,            % na podanej stronie
   bookmarksopen=true,        % rozwinięcie zakładek
   bookmarksopenlevel=1,      % do jakiego poziomu
   colorlinks=true,      % kolorowanie odnośników zamiast ramki wokół nich
   citecolor=cyan,       % kolor odnośników do bibliografii, domyślnie zielony
   filecolor=red,        % kolor odnośników do lokalnych plików, domyśnie magenta
   linkcolor=blue,       % kolor odnośników wewnętrznych, domyślnie czerwony
   menucolor=green,      % kolor pozycji menu Acrobata, domyślnie czerwony
   urlcolor=blue,        % kolor odnośników do adresów internetowych, domyślnie cyan
                              % DANE DOKUMENTACJI
   pdftitle={Bąk jaki jest każdy widzi. Studium zachowań - Przykład
    i~wytyczne formatowania pracy dyplomowej},
   pdfauthor={Roberto Orozco, Robert Muszyński},
   pdfsubject={Praca dyplomowa inżynierska - przykład i~wytyczne},
   pdfkeywords={bąk, Lagrange top, Euler top, praca dyplomowa, formatowanie, wytyczne}
}

%-------------------------| Geometria strony |-----------------------------

\geometry{
    top = 25mm,
    headheight = 15mm,
    headsep = 3mm,
    textheight = 24cm,
    textwidth = 16cm,
}

%---------------------| Frequently used commands definitions |-------------------

\newcommand{\red}{\color{red}}
\def\BibTeX{{\rm B\kern-.05em{\sc i\kern-.025em b}\kern-.08em
    T\kern-.1667em\lower.7ex\hbox{E}\kern-.125emX}}

\newtheorem{uwaga}{Uwaga}
\newtheorem{twr}{Twierdzenie}
%------------------------------------| END |-------------------------------------

%----------------------------| Math symbols definition |-------------------------

\newcommand{\angmom}{\boldsymbol{m}}
\newcommand{\bdvelo}{\boldsymbol{\omega}_B}
\newcommand{\grav}{\boldsymbol{g}}
\newcommand{\lagran}{L(\boldsymbol{q},\boldsymbol{\dot{q}})}
\newcommand{\COMvec}{\boldsymbol{r}_B}
\newcommand{\ee}{\boldsymbol{e}}
\newcommand{\FF}{\boldsymbol{F}}
\newcommand{\xx}{\boldsymbol{x}}
\newcommand{\qq}{\boldsymbol{q}}
\newcommand{\RR}{\boldsymbol{R}}
\newcommand{\TT}{\boldsymbol{T}}
\newcommand{\pp}{\boldsymbol{p}}
\newcommand{\iner}{\boldsymbol{I}_B}
\newcommand{\vv}{\boldsymbol{v}}
\newcommand{\bbs}{\boldsymbol}
\DeclareMathOperator{\const}{const}
\DeclareMathOperator*{\rank}{rank} 

\graphicspath{{figures/chapter_01/}{figures/chapter_02/}{figures/chapter_03/}{figures/chapter_04/}}


\begin{document}

\pdfbookmark{Strona tytułowa}{tytul}
\maketitle
%% informacja o sposobie udostępniania tego dokumentu
\thispagestyle{empty}
\mbox{}
\vfill

\noindent
{\bf Robert Muszyński, Roberto Orozco}\\
{\bf Wrocław 2019}\\[2ex]
  \includegraphics[width=0.18\textwidth]{figures/CC-BY-SA_icon_svg.png}\hfill
\begin{minipage}[b]{0.79\textwidth}
 \small Szablon jest dostępny na licencji Creative Commons: \emph{Uznanie au\-tor\-stwa-Na tych samych warunkach 4.0 Polska}
\end{minipage}\vspace{2ex}

\noindent
{\normalsize Utwór udostępniany na licencji Creative Commons: uznanie
  autorstwa, na tych samych warunkach. Udziela się zezwolenia do
  kopiowania, rozpowszechniania i/lub modyfikacji treści utworu
  zgodnie z zasadami w/w licencji opublikowanej przez Creative
  Commons. Licencja wymaga podania oryginalnego autora utworu, a
  dystrybucja materiałów pochodnych może odbywać się tylko na tych
  samych warunkach (nie można zastrzec, w jakikolwiek sposób
  ograniczyć, ani rozszerzyć praw do nich). Tekst licencji jest
  dostępny pod adresem:
  \url{https://creativecommons.org/licenses/by-sa/4.0/legalcode.pl}.}


\cleardoublepage
\pdfbookmark{\contentsname}{Contents}
\tableofcontents            %spis treści
\markboth{\contentsname}{\contentsname}
%%\newpage
%\thispagestyle{empty}
%\cleardoublepage
%\thispagestyle{plain}

\mbox{}\vfill\hfill
\begin{minipage}{0.5\linewidth} 
  {\tiny \noindent Do składu pracy wykorzystano system przygotowania
    dokumentów~\LaTeX, opracowany przez
    L.~Lamporta\index{latex>\LaTeX} [Lam94], będący nakładką
    systemu \TeX, [Knu86a,Knu86b].  Matematyczne czcionki o nazwie
    {AMS Euler}, których używamy w tej pracy, zostały przygotowane
    przez H.\ Zapfa [KZ86], przy współpracy z~D.\ Knuthem i~jego
    studentami, na zlecenie Amerykańskiego Towarzystwa Matematycznego.
    %% Przy wybranej Antykwie Toruńskiej/Półtawskiego odznacz odpowiednio poniższe
    Wybrane czcionki składu tekstu, Antykwa Toruńska [Now97] -- jeden
    %Wybrane czcionki składu tekstu, Antykwa Półtawskiego [Now99] -- jeden
    z~nielicznych krojów pisma zaprojektowany specjalnie dla języka
    polskiego w~sposób uwzględniający jego rytm -- w~odczuciu autora
    doskonale współgrają z~kształtem czcionki {AMS Euler}, pozwalając
    na uzyskanie harmonijnej całości.
    % %% Przy wybranych czcionkach Concrete odznacz poniższe
    % Czcionki składu tekstu, zwane {Concrete Roman} i {Concrete
    %   Italic}, należące do knuthowskiej rodziny czcionek {Computer
    %   Modern}, zostały specjalnie przystosowane do kształtu czcionki
    % {AMS Euler} na potrzeby książki [GKP96].
    % %% Przy wybranych czcionkach URW Palladio odznacz poniższe
    % Czcionka składu tekstu, zwana URW Palladio jest klonem zapfoskiej rodziny
    % czcionek o~nazwie Palatino [LPn05] i~zdaniem autora świetnie współgra
    % z~kształtem czcionki {AMS Euler}.
    Składu bezszeryfowego tekstu maszynowego dokonano z~użyciem
    opracowanej przez R. Leviena czcionki o~nazwie Inconsolata
    [Lev15]\footnote{\red\tiny Chyba warto takie informacje szerzyć}.


\vspace{-4mm}

 \makeatletter
\renewenvironment{thebibliography}[1]
     {%
        \tiny%
      \list{\@biblabel{\@arabic\c@enumiv}}%
           {\settowidth\labelwidth{\@biblabel{#1}}%
\setlength{\itemsep}{2.5mm}
            \leftmargin\labelwidth
            \advance\leftmargin\labelsep
            \@openbib@code
            \usecounter{enumiv}%
            \let\p@enumiv\@empty
            \renewcommand\theenumiv{\@arabic\c@enumiv}}%
      \sloppy\clubpenalty4000\widowpenalty4000%
      \sfcode`\.\@m\vspace{5mm}}
     {\def\@noitemerr
       {\@latex@warning{Empty `thebibliography' environment}}%
      \endlist}

\makeatother

\begin{thebibliography}{Knu86b}

% %% Odmarkować pozycję gdy wybrane czcionki Concrete
% \bibitem[GKP96]{GKP96loc}
% R.~L. Graham, D.~E. Knuth i O.~Patashnik,
% \newblock { Matematyka konkretna}.
% \newblock PWN, Warszawa, 1996.\vspace{-3mm}

\bibitem[Knu86a]{Knuth86loc}
D.~E. Knuth,
\newblock { The \TeX book, volume {A} of Computers and Typesetting}.
\newblock Addison-Wesley, Reading, 1986.\vspace{-3mm}

\bibitem[Knu86b]{Knuth86aloc}
D.~E. Knuth,
\newblock { \TeX: {The} Program, volume {B} of Computers and Typesetting}.
\newblock Addison-Wesley, Reading, 1986.\vspace{-3mm}

\bibitem[KZ86]{KnZa89loc}
D.~E. Knuth i H.~Zapf,
\newblock {AMS} {Euler} --- {A} new typeface for mathematics.
\newblock { Scholary Publishing}, {20}:131--157, 1986.\vspace{-3mm}

\bibitem[Lam94]{Lamport94loc}
L.~Lamport,
\newblock { \LaTeX: A Document Preparation System}.
\newblock Addison-\mbox{-Wesley}, Reading, 1994.\vspace{-3mm}

\bibitem[Lev15]{Levien15loc}
R.~Levien,
\newblock {Inconsolata}.
\newblock \url{https://levien.com/type/myfonts/inconsolata.html}, 2015.\vspace{-3mm}

% %% Odmarkować pozycję przy wybranej czcionce URW Palladio
% \bibitem[LPn05]{LinotypePalatino05loc}
% Linotype Palatino nova: A classical typeface redesigned by Hermann Zapf,
% \newblock Linotype Library GmbH, 2005.\vspace{-2mm}

%% Odmarkować pozycję przy wybranej Antykwie Toruńskiej
\bibitem[Now97]{nowacki97loc}
J.~Nowacki,
\newblock {Antykwa} {Toruńska} -– od początku do końca polska czcionka.
\newblock {\em Biuletyn Polskiej Grupy Użytkowników Systemu \TeX}, 9:26--27,
  \nolinebreak1997.\vspace{-2mm}

% %% Odmarkować pozycję przy wybranej Antykwie Półtawskiego
% \bibitem[Now99]{nowacki99}
% J.~Nowacki,
% \newblock Piórkiem i {MetaPost-em}, czyli {Antykwa} {Półtawskiego}.
% \newblock {\em Biuletyn Polskiej Grupy Użytkowników Systemu \TeX}, 12:49--53,
%   \nolinebreak1999.\vspace{-2mm}

\end{thebibliography}
}
\end{minipage}
       %notka o systemie składu i czcionkach
%\input{sources/font-note}   %przykładowa wersja angielska notki

%% część właściwa dokumentu
\chapter{Wstęp}
\label{wstep}
\chapter{Preliminaria matematyczne}\label{ch:02}

\section{Wprowadzenie}\label{sec:intromat}

Odwrócone wahadło to przykład obiektu robotycznego, który podczas normalnej pracy środek swojej masy ciężkości ma powyżej punktu obrotu, przez co jest układem wysoce niestabilnym. Układ taki charakteryzuje się dwoma punktami swobody:
\begin{itemize}
    \item punkt obrotu pręta
    \item poruszający się wzdłużnie wózek
\end{itemize} 
oraz tylko jednym wejściem sterującym:
\begin{itemize}
    \item prędkość wózka wzdłuż osi X.
\end{itemize}
Problem polega na znalezieniu takiego sterowania wózkiem, aby wahadło było utrzymane w osi pionowej, bez opadania w dół. Sterowanie odbywa się poprzez regulację prędkością wózka. \\

Jeżeli ruch jest opisany współrzędnymi uogólnionymi \(q=(q_1,q_2,...,q_n)^T\), które zdefiniowane są przez położenie liniowe i kątowe, oraz prędkościami uogólnionymi \(\dot{q}=(\dot{q}_1,\dot{q}_2,...,\dot{q}_n)^T \), które są pochodnymi położenia po czasie to lagranżian układu, który jest różnicą energii kinetycznej i potencjalnej można opisać wzorem:
\begin{equation}
    L(q,\dot(q))=K(q,\dot(q))-(q)
\end{equation}
Aby otrzymać równania ruchu układu, należy rozwiązać równania Eulera-Lagrange’a
\begin{equation}
    \frac{d}{dt}\frac{\partial  L}{\partial \dot{q}}-\frac{\partial L}{\partial q}=F,
\end{equation} gdzie \textit{F} to wszystkie siły niepotencjalne, takie jak:
\begin{itemize}
    \item siły tarcia,
    \item siły przyczepności,
    \item siły sterujące.
\end{itemize}
W przypadku, gdy żadne siły niepotencjalnie nie oddziaływują na układ podstawia się  \textit{F = 0} \cite{TchMu18}.

\section{Model matematyczny} \label{sec:modelmat}

Układ odwróconego wahadła został przedstawiony na rysunku \ref{fig:draw}. Rzeczywisty układ wahadła uwzględnia moment bezwładności pręta, który w tym modelu został zastąpiony przez pręt o pomijalnie małej masie i niewielkiej masie \textit{m} na jego końcu. 

\begin{figure}
    \centering
    \includegraphics[scale=0.8]{praca_dyplomowa/figures/pendulum_draw.jpg}
    \caption{Układ odwróconego wahadła}
    \label{fig:draw}
\end{figure}

Współrzędne końca wahadła są opisane przez:
\begin{equation}
    \begin{cases}
    x_{p}=x+L\sin{\Theta}\\ 
    y_{p}=-L\cos{\Theta}
    \end{cases}
\end{equation}
gdzie \textit{x} to składowa położenia masy \textit{M}, wyznaczona przez rzut środka ciężkości na oś X układu. Składowe prędkości masy \textit{m} można wyznaczyć poprzez obliczenie pierwszych pochodnych po czasie jego współrzędnych pierwsze pochodne jego współrzędnych: 
\begin{equation}
    \begin{cases}
    \dot{x}_{p}=\dot{x}+L\dot{\Theta}\cos{\Theta}\\
    \dot{y}_{p}=L\dot{\Theta}\sin{\Theta}
    \end{cases}
    \label{skladowe}
\end{equation}

Energię kinetyczną można przedstawić w postaci równania \ref{ekin}, przy założeniu, że ramię wahadła ma pomijalnie małą masę, a tym samym jego moment bezwładności jest równy 0.
\begin{equation}
    K=\frac{1}{2}M{v_{1}}^{2}+\frac{1}{2}m{v_{2}}^{2},
    \label{ekin}
\end{equation}
gdzie \textit{v\textsubscript{1}} i \textit{v\textsubscript{2}} to odpowiednio prędkości mas \textit{M} i \textit{m} i wynoszą:
\begin{equation}
    \begin{array}{l}
         v_1=\dot{x} \\
         v_2=\sqrt{{\dot{x}_p}^{2}+{\dot{y}_p}^{2}}
    \end{array}
\end{equation}
Korzystając z wyprowadzenia \ref{skladowe} kwadrat prędkość \textit{v\textsubscript{2}} można wyrazić następująco: 
\begin{equation}
    {v_2}^2=(\dot{x}+L\dot{\Theta}\cos{\Theta})^{2}+(L\dot{\Theta}\sin{\Theta})^{2}=\dot{x}^2+2L\dot{\Theta}\dot{x}\cos{\Theta}+L^2\dot{\Theta}^2,
\end{equation}
zatem energia kinetyczna układu jest równa:
\begin{equation}
    K=\frac{1}{2}(M+m)\dot{x}^2+mL\dot{\Theta}\dot{x}\cos{\Theta}+\frac{1}{2}mL^2\dot{\Theta}^2
\end{equation}

,,Energia potencjalna układu pochodzi od siły grawitacji działającej na kulę" \cite{TchMu18} i wyraża się wzorem:
\begin{equation}
    V=mgL\cos{\Theta}
\end{equation}

Łącząc wyprowadzone równania na energię kinetyczną i potencjalną otrzymamy lagranżian:

\begin{equation}
    L=K-V=\frac{1}{2}(M+m)\dot{x}^2+mL\dot{\Theta}\dot{x}\cos{\Theta}+\frac{1}{2}mL^2\dot{\Theta}^2-mgL\cos{\Theta}
\end{equation}

Do wyznaczenia równań Eulera-Lagrange’a potrzebne są następujące pochodne:
\begin{equation}
        \begin{array}{l}
         \frac{\partial L}{\partial x}=0 \\ \\
         \frac{\partial L}{\partial \Theta}=mg\sin{\Theta} \\ \\
         \frac{\partial L}{\partial \dot{x}}=(M+m)\dot{x}+mL\dot{\Theta}\cos{\Theta} \\ \\
         \frac{\partial L}{\partial \dot{\Theta}}=mL\dot{x}\cos{\Theta}+mL^2\dot{\Theta}
    \end{array}
    \label{pochodne}
\end{equation}

Obliczone pochodne \ref{pochodne} należy wykorzystać, aby otrzymać równania dynamiki:
\begin{equation}
    \label{rDyna}
    \begin{cases}
    \frac{d}{dt}\frac{\partial L}{\partial \dot{x}}-\frac{\partial L}{\partial x}= (M+m)\ddot{x}+mL\ddot{\Theta}\cos{\Theta}-mL\dot{\Theta}^2\sin{\Theta}=F-T \\
    \frac{d}{dt}\frac{\partial L}{\partial \dot{\Theta}}-\frac{\partial L}{\partial \Theta}=ml\ddot{x}\cos{\Theta}+mL^2\ddot{\Theta}-mgL\sin{\Theta},
    \end{cases}
\end{equation}
gdzie \textit{F} jest siłą sterująca, natomiast \textit{T} tarciem.
\chapter{Budowa urządzenia}

\section{Konstrukcja}

Część mechaniczna urządzenia została zamodelowana w programie Solidworks, dzięki czemu ustalono jakie elementy konstrukcyjne są potrzebne oraz czy nie występują kolizje z poruszającym się wózkiem. Koncepcyjny model 3D zaprezentowano na rysunku \ref{fig:konstrukcja}. 

\begin{figure}
    \centering
    \includegraphics[scale=0.5]{praca_dyplomowa_wzor/figures/pendulum6.jpg}
    \caption{Poglądowy model 3D konstrukcji}
    \label{fig:konstrukcja}
\end{figure}

\subsection{Rama}
Do budowy konstrukcji urządzenia zostały wykorzystane aluminiowe profile typu V-Slot 2040, jak na rysunku \ref{fig:profil}. Przekrój profilu ma wymiary 20 x 40 mm, a jego boki mają charakterystyczne rowki w kształcie litery V. Rama została zbudowana z 3 profili o długości 500mm oraz 4 profili długości 250mm. Wszystkie elementy zostały ze sobą połączone dedykowanymi elementami wydrukowanymi na drukarce 3D oraz śrubami M4. Część elementów łączeniowych została znaleziona na stronie \textit{thingiverse.com}, na której udostępniane są modele 3D przygotowane z myślą o ich wydrukowaniu. Konstrukcja złożona w ten sposób charakteryzuje się dużą sztywnością, a przy tym niską masą.

\begin{figure}
    \centering
    \includegraphics[scale=0.7]{praca_dyplomowa_wzor/figures/profil.jpg}
    \caption{Aluminiowy profil 2040}
    \texttt{Źródło: v-slot.pl}
    \label{fig:profil}
\end{figure}

\subsection{Układ jezdny}
Poruszający się wózek, popularnie nazywany karetką, został zbudowany na stalowej blasze, do którego przymocowano 4 łożyskowane rolki widoczne na rysunku \ref{fig:Rolka}, które są dedykowane do poruszania się po profilach V-Slot. Dodatkowo dolne rolki zostały zamontowane na tulejach mimośrodowych, dzięki czemu można regulować siłę docisku rolek do profilu. Do karetki przymocowany pas zębaty GT2 o szerokości 6mm, który z jednej strony porusza się po kole pasowym będącym napinaczem, a z drugiej po kole zębatym, które jest zamontowane na wale silnika. 

\begin{figure}
    \centering
    \includegraphics[scale=0.3]{praca_dyplomowa_wzor/figures/wheel.jpg}
    \caption{Rolka jezdna}
    \texttt{Źródło: black-frog.pl}
    \label{fig:Rolka}
\end{figure}

\section{Podzespoły}

\subsection{Mikrokontroler}
W urządzeniu został wykorzystany mikrokontroler firmy STMicroelectronics STM32F407 Discovery ukazany na rysunku \ref{fig:STM32}. Moduł wyposażony jest w 32-bitowy rdzeń ARM Cortex M4F. Maksymalne taktowanie mikroprocesora to 168 MHz, oferuje 1 MB pamięci flash oraz 192 kB pamięci RAM. Dodatkowo moduł jest wyposażony w dedykowany programator ST-LINK/V2, który pozwala również na debuggowanie. Mikrokontroler oferuje wiele programowalnych wejść i wyjść, a w tym między innymi w trybie wejścia sygnału enkodera czy wyjścia sygnału PWM. Moduł można zasilać zarówna z 5V jak i 3,3V. Płytka wraz ze sterownikiem silnika zostały przymocowane do konstrukcji przy pomocy samodzielnie zaprojektowanego uchwytu. 

\begin{figure}
    \centering
    \includegraphics[scale=0.8]{praca_dyplomowa_wzor/figures/STM32F407.jpg}
    \caption{Mikrokontroler STM32F407 Discovery}
    \texttt{Źródło: botland.com.pl}
    \label{fig:STM32}
\end{figure}

\subsection{Enkoder}
Jako czujnik kąta odchylenia wahadła został wykorzystany enkoder DFRobot 400P/R, przedstawiony na rysunku \ref{fig:Enkoder}. Czujnik ma rozdzielczość 400 impulsów na każdy z 2 kanałów. Na wyjściach generowane są sygnały kwadraturowe, które są przesunięte w fazie względem siebie o 90\degree. W zależności od tego, na którym kanale najpierw pojawia się sygnał można rozróżnić, kierunek obrotów. Przy zliczaniu zliczaniu wszystkich zboczy sygnałów maksymalna rozdzielczość wynosi 1600 impulsów na obrót z czego wynika, ze jeden impuls to odchylenie o 0,225\degree. Enkoder może być zasilany napięciem od 4,8V do 24V. Sygnały wyjściowe czujnika są generowane przez tranzystory NPN w układzie otwartego kolektora, przez co linie sygnałowe wymagają podciągnięcia przez rezystory do dodatniej linii zasilania (pull-up).

\begin{figure}
    \centering
    \includegraphics[scale=0.7]{praca_dyplomowa_wzor/figures/encoder.jpg}
    \caption{Enkoder DFRobot 400P/R}
    \texttt{Źródło: jsumo.com}
    \label{fig:Enkoder}
\end{figure}

\subsection{Silnik}
Do poruszania karetką został wykorzystany silnik prądu stałego Pololu 37Dx65L, ukazany nna rysunku \ref{fig:Silnik}. Napięcie zasilania silnika to 12V, minimalny pobór prądu to 0,2A, a maksymalny przy zatrzymaniu wału 5,5A. Silnik wyposażony jest w przekładnię o przełożeniu 6,25:1, dzięki której osiąga 1600 RPM przy nominalnym napięciu, a moment obrotowy wynosi 3kg/cm, czyli 0,294Nm. Dodatkowo silnik posiada własny enkoder o rozdzielczości 16 impulsów na obrót na każdym kanale co maksymalnie daje 64 impulsy na pełny obrót. Enkoder silnik może być zasilany napięciem od 3,5V do 20V. 

\begin{figure}
    \centering
    \includegraphics[scale=0.3]{praca_dyplomowa_wzor/figures/Pololu 37D.jpg}
    \caption{Silnik DC Pololu 37D}
    \texttt{Źródło: kamami.pl}
    \label{fig:Silnik}
\end{figure}

\subsection{Sterownik silnika}
Do sterowania silnikiem DC został wykorzystane moduł wyposażony w sterownik L298N, widoczny na rysunku \ref{fig:L298N}. Układ ten umożliwia sterowanie 2 silnikami dzięki dwóm osobnym kanałom. Napięcie zasilania silników może być dowolne z przedziału 4,8V do 46V, natomiast część logiczna wymaga zasilania napięciem 5V. Maksymalny prąd wyjściowy to 2A na każdy kanał, a dodatkowo kanały można połączyć ze sobą równolegle przez dwukrotnie zwiększy się wydajność prądowa. Moduł, dzięki zastosowaniu mostka H, pozwala sterować silnikiem w różnych kierunkach, a także stosować szybkie lub wolne hamowanie. Prędkość obrotów silnika można regulować sygnałem PWM.

\begin{figure}
    \centering
    \includegraphics[scale=0.2]{praca_dyplomowa_wzor/figures/L298N.JPG}
    \caption{Sterownik silników DC L298N}
    \texttt{Źródło: botland.com.pl}
    \label{fig:L298N}
\end{figure}

\subsection{Zasilacz}
Do zasilania wszystkich podzespołów został użyty zasilacz komputerowy ATX o maksymalnej mocy 400W jak na rysunku \ref{fig:zasilacz}. Zaletą tego rozwiązania jest zapewnienie potrzebnych różnych napięć zasilania, bez potrzeby stosowania dodatkowych przetwornic. Zasilacz został umocowany do ramy przy pomocy samodzielnie zaprojektowanego i wydrukowanemu uchwytu.

\begin{figure}
    \centering
    \includegraphics[scale=0.8]{praca_dyplomowa_wzor/figures/zasilacz.jpg}
    \caption{Zasilacz ATX 400W}
    \texttt{Źródło: internet-chorzow.pl}
    \label{fig:zasilacz}
\end{figure}




\include{sources/50_zakonczenie}

\cleardoublepage
\phantomsection
\addcontentsline{toc}{chapter}{\bibname}
\bibliographystyle{alphapl}
\bibliography{sources/bibliografia}
\markboth{\bibname}{\bibname}

%%spis rysunków
\cleardoublepage
\phantomsection
\addcontentsline{toc}{chapter}{\listfigurename}
\listoffigures
\markboth{\listfigurename}{\listfigurename}

\appendix
\chapter{Skrypt symulacji}
\begin{figure}[tp]
    \centering
    \includegraphics[height=\textheight]{praca_dyplomowa/figures/simpend.png}
    \caption{Główny skrypt symulacji}
    \label{fig:simpend}
\end{figure}

\begin{figure}
    \centering
    \includegraphics[width=\textwidth]{praca_dyplomowa/figures/cartpend.png}
    \caption{Funkcja \textit{cartpend}}
    \label{fig:cartpend}
\end{figure}

\begin{figure}
    \centering
    \includegraphics[width=\textwidth]{praca_dyplomowa/figures/drawpend.png}
    \caption{Funkcja \textit{drawpend}}
    \label{fig:drawpend}
\end{figure}

\begin{figure}
    \centering
    \includegraphics[height=\textheight]{praca_dyplomowa/figures/mypid.png}
    \caption{Funkcja \textit{mypid}}
    \label{fig:mypid}
\end{figure}

\begin{figure}
    \centering
    \includegraphics[width=\textwidth]{praca_dyplomowa/figures/pidkod.png}
    \caption{Algorytm regulatora \textit{PID IND}}
    \label{fig:pidind}
\end{figure}

\end{document}
